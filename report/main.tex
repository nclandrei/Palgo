\documentclass[11pt]{article}

\usepackage[a4paper,margin=20mm]{geometry}
\usepackage{titlesec}
\usepackage{graphicx}
\usepackage{titling}
\usepackage{wrapfig}

\setlength{\droptitle}{-19mm}
\posttitle{\begin{center}\par\end{center}\vspace{-5mm}}
\fboxsep=10mm%padding thickness
\fboxrule=0pt%border thickness
\renewcommand\abstractname{Overview}

\begin{document}

\pretitle{%
  \begin{center}
  \LARGE
  \includegraphics[scale=0.75]{CompSci_mono.pdf}\\[\bigskipamount]
}
\posttitle{\end{center}}

\title{Level 4 Project - Algorithm Animator \\ Status Report}
\author{Andrei-Mihai Nicolae (2147392) \\ Supervisor: Dr. Gethin Norman}

\maketitle

\section{Project Description}
 The project is aimed at students who want to understand algorithms better - it will help them visualize what is going under the hood of some algorithms taught in Algorithmics 3 class. The end product should be a user friendly cross-platform application (i.e. Linux, MacOS, Windows) which should show animations corresponding to steps in performing the algorithm on user-defined input.
 
 \section{Progress}
 After discussing different approaches with Dr. Norman, I have decided to use the Electron framework (i.e. web-based framework built by GitHub; it is used to power their own text editor, Atom, as well as many other desktop applications). This is the core of the project and, as mentioned, it uses web tools: HTML, CSS, JavaScript. Because web-based apps have such success in the technology world, the multitude of open source libraries and support would provide enough opportunities to create both a user-friendly and efficient application that would run on any major operating system (and feel native at the same time).
 
 In terms of algorithms implemented, I have started with the Huffman Tree and I can say that this major step is almost finished. Even though I tried different JavaScript graph representation libraries (i.e. TreantJS, SigmaJS, Vis.js), I decided to use vis.js in the end as it provides a huge amount of features and very well written documentation. The user has the chance of choosing how to input the text: input it manually in a text box, randomly generated, as well as reading from a file. 
 
 After this algorithm is fully implemented (which is aimed to be done before Christmas), I will proceed with other algorithms as well (after discussions with Dr. Norman, we have thought of maybe digging into graphs). The only step that needs to be completed is making the animations look more user-friendly.
 
 In terms of design, I have decided to use Material Design from Google as this is one of the newest web standards in terms of styling, trying to apply most of the HCI techniques I learned during the classes we had in the past 3 years.
 
 \section{Plan}
 A rough plan would be as follows:
 \begin{itemize}
 	\item Before Christmas: finish the Huffman Tree algorithm and try to add the final design features of the app
 	\item By mid January: add AngularJS, TypeScript and (maybe) ReactJS in order to make it more scalable and readable; this will allow future developers to easily implement extra algorithms on the platform
 	\item By end of February: Add at least 3 more algorithms as they would be easier to implement (the skeleton of the app I made will allow modular implementation, thus it will significantly speed up the process)
 	\item Before project submission deadline: finish the report test the app on all platforms in order to ensure correctness and efficiency (I will try to organize focus groups with 3rd year students)
 \end{itemize}
 
\section{Problems}
One of the biggest problems I faced right from the beginning was the fact that most algorithm animators already implemented (i.e. either by developers open-sourcing them or former Glasgow University CS students) were written using Java FX or Swing. I thought that this approach would not feel native on the user's computer, hence I wanted to use some web framework and use the huge ecosystem that surrounds JavaScript. Therefore, I had some issues in finding useful resources to help me see how some other animators were implemented using web tools.

One other problem I had was the main engine of the app, which is the graph representation library. I had to find a JavaScript library that would allow the project to be flexible, customizable and scalable. I tried, as mentioned before, more than 3 different libraries and even though I dived into each one of them and made progress, I was not satisfied with any of them until I found Vis.js. This is the one I chose to use eventually.

Last but not least, I ran into issues with finding some good design tools. Because there is such a multitude of possibilities, I had plenty to choose from. However, after talking to some of my classmates requesting feedback on different design approaches, I decided that the material one would be the perfect fit for the project.

\end{document}
